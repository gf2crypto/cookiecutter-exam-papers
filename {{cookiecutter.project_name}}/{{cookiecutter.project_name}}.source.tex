% !TEX root=./{{cookiecutter.project_name}}.a6paper.tex

\begin{questions}
    \question
        Неоднородная модель квантовых вычислений. Квантовый алгоритм Дойча--Йожи
    \question
        Задача о тройкосочетаниях, её \(\NP\)-полнота.
    \question
        Помогите Анне расшифровать сообщение \(0x53,\) зашифрованное Борисом криптосистемой Мак-Элиса, построенной на основе кодов Хэмминга \(\mathcal{H}_3\), если её секретный ключ равен 
        \[\{[12, 6, 9, 11]; (4, 5, 1, 3, 6, 2, 0)\}.\]
        \emph{Пояснение: матрица задана списком (в квадратных скобках) целых чисел, двоичное представление которых равно строкам матрицы; подстановка (в круглых скобках) задана обычным образом.}

\end{questions}


\begin{questions}
    \question
        Постквантовые криптографические алгоритмы, построенные на основе хэш-функций.
        Алгоритм электронной цифровой подписи Меркля.
        Дерево Меркля.
    \question
        \(\NP\)-полнота задачи декодирования кода общего положения и \(\NP\)-полнота задачи о весовом спектре кодов.
    \question
        Помогите Анне расшифровать сообщение \(0x1E,\) зашифрованное Борисом криптосистемой Мак-Элиса, построенной на основе кодов Хэмминга \(\mathcal{H}_3\), если её секретный ключ равен 
        \[\{[3, 10, 12, 13]; (3, 6, 0, 1, 2, 4, 5)\}.\]
        \emph{Пояснение: матрица задана списком (в квадратных скобках) целых чисел, двоичное представление которых равно строкам матрицы; подстановка (в круглых скобках) задана обычным образом.}
\end{questions}


\begin{questions}
    \question
        Постквантовые криптографические алгоритмы, построенные на основе кодов, исправляющих ошибки.
        Базовые понятия теории кодов, исправляющих ошибки.
        Сложные задачи теории кодов, исправляющих ошибки.
    \question
        Задача о тройкосочетаниях, её \(\NP\)-полнота.
    \question
        Помогите Анне расшифровать сообщение \(0x6d,\) зашифрованное Борисом криптосистемой Мак-Элиса, построенной на основе кодов Хэмминга \(\mathcal{H}_3\), если её секретный ключ равен 
        \[\{[9, 7, 6, 5]; (2, 4, 1, 6, 0, 3, 5)\}.\]
        \emph{Пояснение: матрица задана списком (в квадратных скобках) целых чисел, двоичное представление которых равно строкам матрицы; подстановка (в круглых скобках) задана обычным образом.}
\end{questions}


\begin{questions}
    \question
        Постквантовые криптографические алгоритмы, построенные на основе кодов, исправляющих ошибки.
        Базовые понятия теории кодов, исправляющих ошибки.
        Сложные задачи теории кодов, исправляющих ошибки.
    \question
        Эквивалентность кодов.
        Эквивалентность графов.
        Связь задачи эквивалентности кодов с задачей эквивалентности графов.
    \question
        Помогите Семёну-Редиске восстановить секретный ключ Анны, если она пользуется криптосистемой Мак-Элиса, построенной на основе кодов Рида--Маллера первого порядка \(RM(1, 4)\), и её открытый ключ равен 
        \[[59154, 53972, 16859, 43095, 61793].\]
        \emph{Пояснение: матрица задана списком (в квадратных скобках) целых чисел, двоичное представление которых равно строкам матрицы.}

\end{questions}


\begin{questions}
    \question
        Неоднородная модель квантовых вычислений. Квантовый алгоритм Дойча--Йожи
    \question
        \(\NP\)-полнота задачи декодирования кода общего положения и \(\NP\)-полнота задачи о весовом спектре кодов.
    \question
        Помогите Семёну-Редиске дешифровать сообщение \(0x38,\) зашифрованное Борисом криптосистемой Мак-Элиса, построенной на основе кодов Хэмминга \(\mathcal{H}_3\), для Анны, если её открытый ключ равен 
        \[[52, 75, 23, 82].\]
        \emph{Пояснение: матрица задана списком (в квадратных скобках) целых чисел, двоичное представление которых равны строкам матрицы.}
\end{questions}


\begin{questions}
    \question
        Эквивалентность кодов.
        Понятие протокола интерактивного доказательства.
        Протокол Артура--Мерлина.
        Протокол интерактивного доказательства для задачи неэквивалентности кодов.
        Следствие из существования такого протокола.
    \question
        Строение группы автоморфизмов кода Рида--Маллера первого порядка.
    \question
        Помогите Анне расшифровать сообщение \(0x2A6C,\) зашифрованное Борисом криптосистемой Мак-Элиса, построенной на основе кодов Рида--Маллера первого порядка \(RM(1,4)\), если её секретный ключ равен \[\{[12, 18, 26, 15, 14]; (12, 15, 1, 5, 3, 9, 10, 2, 0, 4, 14, 7, 6, 11, 8, 13)\}.\]
        \emph{Пояснение: матрица задана списком (в квадратных скобках) целых чисел, двоичное представление которых равно строкам матрицы; подстановка (в круглых скобках) задана обычным образом.}
\end{questions}



\begin{questions}
    \question
        Криптосистема Мак-Элиса, построенная на основе произвольного класса кодов.
    \question
        Эквивалентность кодов.
        Эквивалентность графов.
        Связь задачи эквивалентности кодов с задачей эквивалентности графов.
    \question
        Помогите Анне расшифровать сообщение \(0xCF5E,\) зашифрованное Борисом криптосистемой Мак-Элиса, построенной на основе кодов Рида--Маллера первого порядка \(RM(1,4)\), если её секретный ключ равен \[\{[5, 15, 18, 14, 27]; (10, 3, 14, 0, 7, 15, 8, 6, 5, 9, 12, 4, 13, 11, 1, 2)\}.\]
        \emph{Пояснение: матрица задана списком (в квадратных скобках) целых чисел, двоичное представление которых равно строкам матрицы; подстановка (в круглых скобках) задана обычным образом.}
\end{questions}


\begin{questions}
    \question
        Криптосистема Мак-Элиса, построенная на основе произвольного класса кодов.
    \question
        Коды Рида--Маллера первого порядка.
        Алгоритм их быстрого декодирования, его сложность.
    \question
        Помогите Семёну-Редиске восстановить секретный ключ Анны, если она пользуется криптосистемой Мак-Элиса, построенной на основе кодов Хэмминга \(\mathcal{H}_3\), и её открытый ключ равен 
        \[[14, 77, 84, 43].\]
        \emph{Пояснение: матрица задана списком (в квадратных скобках) целых чисел, двоичное представление которых равно строкам матрицы.}
\end{questions}


\begin{questions}
    \question
        Эквивалентность кодов.
        Понятие протокола интерактивного доказательства.
        Протокол Артура--Мерлина.
        Протокол интерактивного доказательства для задачи неэквивалентности кодов.
        Следствие из существования такого протокола.
    \question
        Криптосистема Мак-Элиса, построенная на кодах Хэмминга.
        Атака декодирования на эту криптосистему.
    \question
        Помогите Семёну-Редиске восстановить секретный ключ Анны, если она пользуется криптосистемой Мак-Элиса, построенной на основе кодов Рида--Маллера первого порядка \(RM(1, 4)\), и её открытый ключ равен 
        \[[54595, 11535, 1971, 20797, 6763].\]
        \emph{Пояснение: матрица задана списком (в квадратных скобках) целых чисел, двоичное представление которых равно строкам матрицы.}
\end{questions}


\begin{questions}
    \question
        Неоднородная модель квантовых вычислений. Квантовый алгоритм Дойча--Йожи.
    \question
        Коды Рида--Маллера первого порядка.
        Алгоритм их быстрого декодирования, его сложность.
    \question
        Помогите Семёну-Редиске восстановить секретный ключ Анны, если она пользуется криптосистемой Мак-Элиса, построенной на основе кодов Хэмминга \(\mathcal{H}_3\), и её открытый ключ равен 
        \[[41, 69, 14, 86].\]
        \emph{Пояснение: матрица задана списком (в квадратных скобках) целых чисел, двоичное представление которых равно строкам матрицы.}
\end{questions}


\begin{questions}
    \question
        Криптосистема Мак-Элиса, построенная на основе произвольного класса кодов.
    \question
        Понятие группы автоморфизмов кода.
        Изоморфизм группы автоморфизмов кода и подгруппы линейной группы \(GL_k\) (матрицы, задающие автоморфизм кода).
    \question
        Помогите Анне расшифровать сообщение \(0x6744,\) зашифрованное Борисом криптосистемой Мак-Элиса, построенной на основе кодов Рида--Маллера первого порядка \(RM(1,4)\), если её секретный ключ равен \[\{[13, 26, 3, 11, 7]; (7, 6, 1, 15, 11, 5, 2, 10, 0, 4, 13, 14, 12, 3, 8, 9)\}.\]
        \emph{Пояснение: матрица задана списком (в квадратных скобках) целых чисел, двоичное представление которых равно строкам матрицы; подстановка (в круглых скобках) задана обычным образом.}
\end{questions}


\begin{questions}
    \question
        Неоднородная модель квантовых вычислений. Квантовый алгоритм Дойча--Йожи
    \question
        Строение группы автоморфизмов кода Рида--Маллера первого порядка.
    \question
        Помогите Семёну-Редиске дешифровать сообщение \(0x18,\) зашифрованное Борисом криптосистемой Мак-Элиса, построенной на основе кодов Хэмминга \(\mathcal{H}_3\), для Анны, если её открытый ключ равен 
        \[[108, 54, 37, 99].\]
        \emph{Пояснение: матрица задана списком (в квадратных скобках) целых чисел, двоичное представление которых равно строкам матрицы.}
\end{questions}


\begin{questions}
    \question
        Криптосистема Мак-Элиса, построенная на фиксированном коде.
        Понятие эквивалентности секретных ключей.
        Структура класса эквивалентности секретного ключа криптосистемы Мак-Элиса.
    \question
        Задача о тройкосочетаниях, её \(\NP\)-полнота.
    \question
        Помогите Семёну-Редиске восстановить секретный ключ Анны, если она пользуется криптосистемой Мак-Элиса, построенной на основе кодов Хэмминга \(\mathcal{H}_3\), и её открытый ключ равен 
        \[[15, 69, 89, 53].\]
        \emph{Пояснение: матрица задана списком (в квадратных скобках) целых чисел, двоичное представление которых равно строкам матрицы.}
\end{questions}


\begin{questions}
    \question
        Криптосистема Мак-Элиса, построенная на кодах Рида--Маллера первого порядка.
        Структура класса эквивалентности криптосистемы Мак-Элиса, построенной на кодах Рида--Маллера первого порядка.
    \question
        \(\NP\)-полнота задачи декодирования кода общего положения и \(\NP\)-полнота задачи о весовом спектре кодов.
    \question
        Помогите Семёну-Редиске восстановить секретный ключ Анны, если она пользуется криптосистемой Мак-Элиса, построенной на основе кодов Рида--Маллера первого порядка \(RM(1, 4)\), и её открытый ключ равен 
        \[[2271, 7077, 11580, 20200, 49593].\]
        \emph{Пояснение: матрица задана списком (в квадратных скобках) целых чисел, двоичное представление которых равно строкам матрицы.}
\end{questions}


\begin{questions}
    \question
        Эквивалентность кодов.
        Понятие протокола интерактивного доказательства.
        Протокол Артура--Мерлина.
        Протокол интерактивного доказательства для задачи неэквивалентности кодов.
        Следствие из существования такого протокола.
    \question
        Структурные атаки на криптосистему Мак-Элиса.
        Структурная атака на криптосистему Мак-Элиса, построенную на кодах Рида--Маллера первого порядка
    \question
        Помогите Анне расшифровать сообщение \(0x5CCE,\) зашифрованное Борисом криптосистемой Мак-Элиса, построенной на основе кодов Рида--Маллера первого порядка \(RM(1,4)\), если её секретный ключ равен \[\{[23, 30, 7, 31, 26]; (2, 3, 13, 8, 15, 1, 7, 0, 9, 12, 14, 10, 11, 6, 5, 4)\}.\]
        \emph{Пояснение: матрица задана списком (в квадратных скобках) целых чисел, двоичное представление которых равно строкам матрицы; подстановка (в круглых скобках) задана обычным образом.}
\end{questions}


\begin{questions}
    \question
        Структурные атаки на криптосистему Мак-Элиса.
        Структурная атака на криптосистему Мак-Элиса, построенную на кодах Хэмминга.
    \question
        Строение группы автоморфизмов кода Рида--Маллера первого порядка.
    \question
        Помогите Анне расшифровать сообщение \(0x8C71,\) зашифрованное Борисом криптосистемой Мак-Элиса, построенной на основе кодов Рида--Маллера первого порядка \(RM(1,4)\), если её секретный ключ равен \[\{[18, 22, 8, 3, 10]; (12, 15, 4, 10, 11, 13, 0, 3, 7, 9, 1, 14, 5, 2, 8, 6)\}.\]
        \emph{Пояснение: матрица задана списком (в квадратных скобках) целых чисел, двоичное представление которых равно строкам матрицы; подстановка (в круглых скобках) задана обычным образом.}
\end{questions}


\begin{questions}
    \question
        Неоднородная модель квантовых вычислений. Квантовый алгоритм Дойча--Йожи.
    \question
        Структурные атаки на криптосистему Мак-Элиса.
        Структурная атака на криптосистему Мак-Элиса, построенную на кодах, дуальных к коду Рида--Маллера первого порядка.
    \question
        Помогите Семёну-Редиске дешифровать сообщение \(0x13,\) зашифрованное Борисом криптосистемой Мак-Элиса, построенной на основе кодов Хэмминга \(\mathcal{H}_3\), для Анны, если её открытый ключ равен 
        \[[100, 120, 50, 53].\]
        \emph{Пояснение: матрица задана списком (в квадратных скобках) целых чисел, двоичное представление которых равно строкам матрицы.}
\end{questions}


\begin{questions}
    \question
        Постквантовые криптографические алгоритмы, построенные на основе кодов, исправляющих ошибки.
        Базовые понятия теории кодов, исправляющих ошибки.
        Сложные задачи теории кодов, исправляющих ошибки.
    \question
        Структурные атаки на криптосистему Мак-Элиса.
        Структурная атака на криптосистему Мак-Элиса, построенную на кодах, дуальных к кодам Хэмминга.
    \question
        Помогите Анне расшифровать сообщение \(0xAE56,\) зашифрованное Борисом криптосистемой Мак-Элиса, построенной на основе кодов Рида--Маллера первого порядка \(RM(1,4)\), если её секретный ключ равен \[\{[17, 8, 24, 21, 27]; (7, 9, 4, 13, 2, 8, 1, 0, 12, 6, 3, 5, 10, 11, 14, 15)\}.\]
        \emph{Пояснение: матрица задана списком (в квадратных скобках) целых чисел, двоичное представление которых равно строкам матрицы; подстановка (в круглых скобках) задана обычным образом.}
\end{questions}
